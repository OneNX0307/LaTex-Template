%preamble.tex
%-----------------------------------------------------%
\usepackage{xeCJK}
\usepackage{amsmath,amssymb}
\usepackage{graphicx}
\linespread{1.25}\selectfont
\usepackage{indentfirst}
	\setlength{\parindent}{2em}
\setlength{\parskip}{1em}
\usepackage[left=3.18cm,right=3.18cm,top=2.54cm,bottom=2.54cm]{geometry}
%----------------------代码-------------------------%
\usepackage{listings}
	\lstset{
			%行号
			numbers=left,
			%背景框
			framexleftmargin=10mm,
			frame=none,
			%背景色
			%backgroundcolor=\color[rgb]{1,1,0.76},
			backgroundcolor=\color[RGB]{245,245,244},
			%样式
			keywordstyle=\bf\color{blue},
			identifierstyle=\bf,
			numberstyle=\color[RGB]{0,192,192},
			commentstyle=\it\color[RGB]{0,96,96},
			stringstyle=\rmfamily\slshape\color[RGB]{128,0,0},
			%显示空格
			showstringspaces=false}
%--------------------------------------------------%

%--------------------------页眉页脚-------------------------------%
\usepackage{fancyhdr}
	\pagestyle{fancy}
	\fancyhead{}
  \fancyhead[RO]{\leftmark}
  \fancyhead[LE]{\rightmark}
	\fancyfoot[C]{\thepage}
	\renewcommand{\headrulewidth}{0.4pt}
	\renewcommand{\footrulewidth}{0.4pt}
  \renewcommand{\chaptermark}[1]{\markboth{第\thechapter 章\qquad #1}{}}
%----------------------------------------------------------------%
\renewcommand{\today}{\number\year 年\number\month 月\number\day 日}
\renewcommand{\contentsname}{目 \quad 录}
\usepackage{titlesec}
    \titleformat{\chapter}{\centering\Huge\bfseries}{第\,\thechapter\,章}{1em}{}
%---------------汉化参考文献--------------------%
%\renewcommand{\abstractname}{摘 \quad 要}  %如果是article类用\refname;如果是book类用\bibname
%\renewcommand{\refname}{参考文献}
%\renewcommand{\bibname}{参考文献}
%----------------中英文摘要---------------------%
\newcommand{\enabstractname}{Abstract}
\newcommand{\cnabstractname}{摘要}
\newenvironment{enabstract}{%英文摘要
	\par\small
    \noindent\mbox{}\hfill{\bfseries \enabstractname}\hfill\mbox{}\par
    \vskip 2.5ex}{\par\vskip 2.5ex}
\newenvironment{cnabstract}{%中文摘要
    \par\small
    \noindent\mbox{}\hfill{\bfseries \cnabstractname}\hfill\mbox{}\par
    \vskip 2.5ex}{\par\vskip 2.5ex}
%---------------------------------------------%
\newcommand{\upcite}[1]{\textsuperscript{\cite{#1}}}%参考文献在右上角
\bibliographystyle{plain}
%------------汉化图表标题--------%
\usepackage{caption}
\renewcommand\figurename{\hei 图}
\renewcommand\tablename{\hei 表}
%-----------自动生成书签------------%
\usepackage[bookmarks=true]{hyperref}
%------------------------字体---------------%
%-------------windows----------------------%
%\usepackage{fontspec,xunicode,xltxtra}
%\newfontfamily\youyuan{YouYuan}
%\newfontfamily\hwcaiyun{STCaiyun}
%\newfontfamily\hwhupo{STHupo}
%\newfontfamily\yaoti{FZYaoTi}
%\newfontfamily\kaiti{KaiTi_GB2312}
%\newfontfamily\xsong{NSimSun}
%\newfontfamily\hwsong{STSong}
%\newfontfamily\yahei{Microsoft YaHei}
%\newfontfamily\songti{SimSun}
%\newfontfamily\hwfangsong{STFangsong}
%\newfontfamily\weiti{STXinwei}
%\newfontfamily\heiti{SimHei}
%\newfontfamily\hwxingkai{STXingkai}
%\newfontfamily\hwlishu{STLiti}
%\newfontfamily\zhongsong{STZhongsong}
%\newfontfamily\shuti{FZShuTi}
%\newfontfamily\hwhei{STXihei}
%\newfontfamily\lishu{LiSu}
%\newfontfamily\hwkai{STKaiti}
%\renewcommand{\baselinestretch}{1.25}

%----------------ubuntu-----------------------%
\usepackage{fontspec,xunicode,xltxtra} 
\newfontfamily\songti{simsun.ttc}			%宋体
\newfontfamily\kaiti{kaiti_GB2312.ttf} 		%楷体
\newfontfamily\heiti{simhei.ttf}			%黑体
\newfontfamily\yahei{msyh.ttc}				%微软雅黑
\newfontfamily\times{times.ttf} 			%Times New Romans
%--------------------------------------字号---------------------------------%
\usepackage{fontspec,xunicode,xltxtra}
\newcommand{\chuhao}{\fontsize{42pt}{\baselineskip}\selectfont}     %初号
\newcommand{\xiaochuhao}{\fontsize{36pt}{\baselineskip}\selectfont} %小初号
\newcommand{\yihao}{\fontsize{28pt}{\baselineskip}\selectfont}      %一号
\newcommand{\erhao}{\fontsize{21pt}{\baselineskip}\selectfont}      %二号
\newcommand{\xiaoerhao}{\fontsize{18pt}{\baselineskip}\selectfont}  %小二号
\newcommand{\sanhao}{\fontsize{15.75pt}{\baselineskip}\selectfont}  %三号
\newcommand{\sihao}{\fontsize{14pt}{\baselineskip}\selectfont}      %四号
\newcommand{\xiaosihao}{\fontsize{12pt}{\baselineskip}\selectfont}  %小四号
\newcommand{\wuhao}{\fontsize{10.5pt}{\baselineskip}\selectfont}    %五号
\newcommand{\xiaowuhao}{\fontsize{9pt}{\baselineskip}\selectfont}   %小五号
\newcommand{\liuhao}{\fontsize{7.875pt}{\baselineskip}\selectfont}  %六号
\newcommand{\qihao}{\fontsize{5.25pt}{\baselineskip}\selectfont}    %七号
